%%%%%%%%%%%%%%%%%%%%%%%%%%%%%%%%%%%%%%%%%%%%%%%%%%%%%%%%%%%%%%%%%%%%%%%%%%%%%%%%%%%%%%%%%%%%%%%%%%%%%%%%%%%%
\section{Related work}
%%%%%%%%%%%%%%%%%%%%%%%%%%%%%%%%%%%%%%%%%%%%%%%%%%%%%%

\paragraph*{Sketch-Based Shape Retrieval.} With the rapid growth of 3D shape data, fast and convenient content-based shape retrieval techniques \cite{asurveyofcontentbasedtangeldermultimedia2008} have become important. Content-based methods usually require a user to provide a 3D shape as a query, which introduces a circular dependency in a 3D modeling scenario. In contrast, sketch-based methods \cite{asearchenginefunkhousertog2003,discriminativesketchbasedshaotianjiacgf2011,EitRicBouHilAle12} allow the user to roughly draw the outline of the desired shape from one or more viewpoints in 2D. This is typically a more intuitive and convenient way to describe the user's intention.

Lee and Funkhouser~\shortcite{sketchbasedcompositionfunkhousersbim2008} develop a sketch-based system for retrieving parts from a part database and incorporating them into a 3D model. Xie et al.~\shortcite{sketchtodesignxukaicgf2013} propose a similar system for shape editing by context-aware part replacement. These systems assume the parts are pre-generated by automatic segmentation, and hence retrieval reduces to global 2D-to-3D matching. The systems do not allow the user to generate new parts with novel cuts, or retrieve groups of pre-existing parts with one sketch. In contrast, our approach supports arbitrary cuts of exemplar shapes, driven by the sketch. Further, because shapes are not pre-segmented, we must do partial matching from 2D to 3D at interactive rates, which is a significant technical challenge.

%%%%%%%%%%%%%%%%%%%%%%%%%%%%%%%%%%%%%%%%%%%%%%%%%%%%%%

\paragraph*{Shape Segmentation.} Segmenting shapes into meaningful parts is a fundamental problem in many computer graphics tasks and applications, and both automatic and manual approaches have been developed to tackle this problem \cite{surveyonmeshsegmentationshamircgf2008,benchmarkforsegmentationchenxiaobaisg2009}. Recently, several groups have explored methods for jointly segmenting a set of shapes~\cite{learning3dmeshsegmentationkalosg2010,cosegmentationof3dshapesliuligangcgf2012,jointshapesegmentationhuangqixingsg2011,SidKaiKleZhaCoh11}. By considering a set of shapes as a whole, the segmentation can exploit shared structure and hence yield more coherent results. These works greatly benefit exploratory modeling systems by providing an automatically pre-segmented shape database for part suggestion and re-assembly.

In contrast to these systems, our approach does not assume a pre-segmented and pre-labeled shape database. Instead, we retrieve and segment shapes in real time based on a user sketch. Since, we cannot anticipate the exact boundary of the sketch, an on-the-fly and contour-aware segmentation method is required.

In contrast to sketch-based mesh cutting methods \cite{FanMeshCutting2012}, we do not know the source shape in advance, and we do not sketch directly on the source shape. Instead, we perform 2D-3D partial matching to automatically retrieve database shapes matching the query sketch.

%%%%%%%%%%%%%%%%%%%%%%%%%%%%%%%%%%%%%%%%%%%%%%%%%%%%%%

\paragraph*{Exploratory Shape Modeling.} The idea of incorporating creativity-inspiring exemplar elements into conventional conceptual modeling has been an active topic for the past few years.
%Generally, the core of creativity-supported modeling methods is to offer users the ability to efficiently explore all possible solutions.
Lee et al. \shortcite{LeeExampleGalleries2010} examine the efficicacy of using galleries of examples for creativity support during the design process.
%Talton et al.~\shortcite{TalGibYanHanKol09} described a data-driven approach to support the open-ended modeling of parametric shapes.
Chaudhuri et al.~\shortcite{datadrivenvladenaisa2010} mine a 3D shape database to suggest components to creatively extend a base shape, based on geometric compatibility. In subsequent work, the authors develop a statistical \st{models} \hl{model} of shape semantics to improve the suggestions~\cite{probabilisticreasoningvladlensg2011}. These works build upon the Modeling by Example system of Funkhouser et al.~\shortcite{modelingbyexamplefunkhousersg2004}, which allowed users to query shape databases with 3D proxies for novel parts.

%In parallel work, researchers have worked on automating the shape generation process, so the user directly explores a space of complete shapes. Talton et al.~\shortcite{TalGibYanHanKol09} describe a data-driven approach for exploring a high-dimensional space of parametric shapes. Xu et al.~\shortcite{setevolutionxukaisg2012} present a fit-and-diverse method for evolving a set of assembled shapes to match user intent. Kalogerakis et al.~\shortcite{aprobabilisticmodelvladlensg2012}, Averkiou et al.~\shortcite{AverkiouShapeSynth2014} and others describe generative probabilistic priors for shape synthesis.
%
%Our approach lies in the category of part-suggestion methods. However, the major technical contribution of our work is that we do not rely on a pre-segmented (or pre-labeled) database, and can generate arbitrarily shaped segments to perfectly match user sketches. Hence, our approach allows users to efficiently query a much larger design space. Our work also stands in contrast to Modeling by Example, since we construct proxies by 2D sketching, extract parts automatically, and do not require exemplar shapes to be co-aligned.
